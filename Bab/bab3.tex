\chapter{Analisis}
\label{chap:analisis}

\section{Deskripsi Masalah}
\label{sec:deskripsimasalah}

Pada penilitian ini, akan dibuat suatu aplikasi yang dapat membantu suatu rumah tangga dalam pengelolaan keuangan mereka. Aplikasi ini dapat digunakan oleh setiap anggota rumah tangga untuk mencatat semua transaksi yang mereka lakukan baik pengeluaran maupun pendapatan. Aplikasi ini juga dapat menampilkan laporan sesuai dengan transaksi yang telah tercatat.

Aplikasi ini sendiri terbagi menjadi dua bagian yaitu aplikasi \textit{end-user} yang digunakan langsung oleh para anggota rumah tangga dan aplikasi yang digunakan oleh admin untuk mengelolah data-data aplikasi.

Data-data yang tercatat tentunya akan disimpan kedalam sebuah basis data sehingga aplikasi ini sendiri akan berkomunikasi dengan \textit{server} yang berfungsi sebagai penyimpanan dan pengolahan data yang dibangun diatas \textit{framework} Hadoop. Untuk komunkasi aplikasi dan \textit{server} akan menggunakan HTTP dimana aplikasi akan mengakses \textit{webservice} yang telah disediakan oleh \textit{server}.

\section{\textit{Cloud Computing Model} untuk kasus pembukuan}

\section{Analisis Kebutuhan Perangkat Lunak}

Pada sub-bab ini akan dibahas fitur-fitur yang disediakan aplikasi dan \textit{server}.

\subsection{Fitur Pada Aplikasi \textit{Mobile Device}}
\subsection{Fitur Pada Aplikasi \textit{Website}}
\subsection{Fitur Pada \textit{Server}}