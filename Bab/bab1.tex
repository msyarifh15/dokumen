\chapter{Pendahuluan}
\label{chap:pendahuluan}

\section{Latar Belakang}
\label{sec:latarbelakang}
Maraknya penggunaan perangkat \textit{mobile} dan internet memberikan kemudahan dalam memanfaatkan aplikasi kolaborasi dan komunitas. Hal ini lebih dikuatkan lagi dengan adanya aplikasi jejaring sosial yang menyediakan fasilitas kolaborasi berbasis \textit{cloud}. Dengan berbasis \textit{cloud} pertumbuhan data yang sangat pesat dari berjuta pengguna tidak lagi menjadi masalah dan menimbulkan kekawatiran dalam menggunakan perangkat \textit{mobile} yang terbatas dalam kapasitas penyimpanan. Dengan demikian, memberikan ruang pertumbuhan yang tidak terbatas akan penggunaan aplikasi-aplikasi yang memberikan layanan kepada publik. %Dengan demikian,.... ini maksudnya apa? gk jelas, lu jelasin lebih lagi, ini orng gk bakal ngerti won.

%kasi dlu masalahnya. Tiba2 muncul "`untuk meningkatkan pengelolaan keuangan..."'

Untuk meningkatkan pengelolaan keuangan rumah tangga sangat dimungkinkan mengembangankan aplikasi dan layanan pembukuan. Layanan pembukuan ini ditujukan kepada rumah tangga dengan struktur utuh multi keluarga. Sebuah keluarga secara utuh terdiri atas ayah (kepala rumah tangga), ibu (ibu rumah tangga) sebagai pengurus rumah tangga. Juga beranggotakan anak-anak, maupun sanak keluarga seperti: orang tua, saudara, dan lainnya.

\section{Rumusan Masalah}
\label{sec:rumusanmalasah}

Rumusan masalah pada penelitian ini adalah:
\begin{enumerate}
\item test
\end{enumerate}

\section{Tujuan}
\label{sec:tujuan}

Tujuan dari pembuatan tugas akhir ini adalah:
\begin{enumerate}
\item Membangun aplikasi
\end{enumerate}

\section{Batasan Masalah}
\label{sec:batasanmasalah}

\section{Metodologi Penelitian}
\label{sec:metodologipenelitian}

\section{Sistematika Pembahasan}
\label{sec:sistematikapembahasan}