\chapter{Pendahuluan}
\label{chap:pendahuluan}

\section{Latar Belakang}
\label{sec:latarbelakang}
Maraknya penggunaan perangkat \textit{mobile} dan internet memberikan kemudahan dalam memanfaatkan aplikasi kolaborasi dan komunitas. Hal ini lebih dikuatkan lagi dengan adanya aplikasi jejaring sosial yang menyediakan fasilitas kolaborasi berbasis \textit{cloud}. Fasilitas \textit{cloud} merupakan pemanfaatan teknologi komputer dan internet. Dengan berbasis \textit{cloud} pertumbuhan data yang sangat pesat dari berjuta pengguna tidak lagi menjadi masalah dan menimbulkan kekawatiran dalam menggunakan perangkat \textit{mobile} yang terbatas dalam kapasitas penyimpanan. Dengan demikian, memberikan ruang pertumbuhan yang tidak terbatas akan penggunaan aplikasi-aplikasi yang memberikan layanan kepada publik. 

Pada saat penelitian ini dilakukan kehidupan rumah tangga moderen tidak lepas dari masalah keuangan. Kepada rumah tangga dan ibu rumah tangga tidak pengetahui pentingnya melakukan pengelolaan keuangan. Ketidak tahuan tersebut dikarenakan pengelolaan laporan keuangan yang mengharuskan pencatatan pengeluaran dan pendapatan dari masing-masing anggota keluarga. Pembuatan laporan keuangan juga tidak mudah sebab memakan waktu dan tenaga. Karena hal tersebut baik kepada rumah tangga maupun ibu rumah tangga cenderung tidak mengetahui keseimbangan pendapatan dan pengeluaran mereka.

%harus ditambahin paragraf yang ngasih tau apa itu layanan pembukuan 

Untuk meningkatkan pengelolaan keuangan rumah tangga sangat dimungkinkan mengembangankan aplikasi dan layanan pembukuan. Layanan pembukuan ini ditujukan kepada rumah tangga dengan struktur utuh multi keluarga. Sebuah keluarga secara utuh terdiri atas ayah (kepala rumah tangga), ibu (ibu rumah tangga) sebagai pengurus rumah tangga. Juga beranggotakan anak-anak, maupun sanak keluarga seperti: orang tua, saudara, dan lainnya. Masing-masing anggota dapat mencatat pengeluaran dan pendapatan masing-masing serta kepala anggota dapat langsung melihat laporan dari semua anggota keluarganya.

\section{Rumusan Masalah}
\label{sec:rumusanmalasah}

Rumusan masalah pada penelitian ini adalah:
\begin{enumerate}
\item Bagaimana merancang aplikasi pembukuan rumah tangga dengan sistem peran?
\item Bagaimana mengimplementasikan aplikasi pada poin nomor satu dengan \textit{framework} Hadoop sehingga aplikasi dapat menyimpan dan mengolah data yang besar dan secara bersamaan?
\end{enumerate}

\section{Tujuan}
\label{sec:tujuan}

Tujuan dari pembuatan tugas akhir ini adalah:
\begin{enumerate}
\item Merancang aplikasi pembukuan rumah tangga dengan sistem peran.
\item Mengimplementasikan aplikasi pada poin nomor satu dengan \textit{framework} Hadoop.
\end{enumerate}

\section{Batasan Masalah}
\label{sec:batasanmasalah}

\section{Metodologi Penelitian}
\label{sec:metodologipenelitian}

Skiprsi ini bersifat deskriptif, yaitu mendeskripsikan data baik dari literatur dan hasil pengujian.

\section{Sistematika Pembahasan}
\label{sec:sistematikapembahasan}

\begin{itemize}
	\item BAB 1 Pendahuluan memuat latar belakang, rumusan masalah, tujuan, ruang lingkup kajian, metode dan teknik pengumpulan data, dan sistematika penulisan.
	\item BAB 2 Dasar Teori memuat teori-teori yang menunjang dalam pembuatan skripsi ini.
	\item BAB 3 Analisis memuat deskripsi masalah, model \textit{cloud computing}, dan analisis perangkat lunak.
	\item BAB 4 Disain memuat disain antar muka, disain basis data, disain aplikasi, dan disain Webservice.
	\item BAB 5 Implementasi dan Pengujian memuat lingkungan implementasi, konfigurasi implementasi, implementasi basis data, implementasi aplikasi, implementasi pengujian fungsional, pengujian eksperimen, dan kesimpulan hasil pengujian.
	\item BAB 6 Kesimpulan dan Saran memuat kesimpulan dan saran yang berdasarkan hasil analisis implementasi dan pengujian.
\end{itemize}
