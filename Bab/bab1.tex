\chapter{Pendahuluan}
\label{chap:pendahuluan}

\section{Latar Belakang}
\label{sec:latarbelakang}


Maraknya penggunaan perangkat mobile dan internet memberikan kemudahan pemanfaatan aplikasi kolaborasi dan komunitas. Hal ini lebih dikuatkan lagi dengan adanya aplikasi jejaring sosial yang menyediakan fasilitas kolaborasi berbasis cloud. Dengan basis cloud pertumbuhan data yang pesat dari berjuta-juta pengguna tidak lagi menjadi masalah dan kekawatiran walau menggunakan perangkat mobile yang terbatas dalam kapasitas penyimpanan. Dengan demikian, memberikan ruang pertumbuhan yang tidak terbatas akan penggunaan aplikasi-aplikasi yang memberikan layanan kepada publik.

Untuk meningkatkan pengelolaan keuangan rumah tangga sangat dimungkinkan mengembangankan aplikasi dan layanan pembukuan. Layanan pembukuan ini ditujukan kepada rumah tangga dengan struktur utuh multi keluarga. Sebuah keluarga secara utuh terdiri atas ayah (kepala rumah tangga), ibu (ibu rumah tangga) sebagai pengurus rumah tangga. Juga beranggotakan anak-anak, maupun sanak famili seperti: orang tua, saudara, dan lainnya.

\section{Rumusan Masalah}
\label{sec:rumusanmalasah}

\section{Tujuan}
\label{sec:tujuan}

\section{Batasan Masalah}
\label{sec:batasanmasalah}

\section{Metodologi Penelitian}
\label{sec:metodologipenelitian}

\section{Sistematika Pembahasan}
\label{sec:sistematikapembahasan}